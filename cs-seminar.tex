\documentclass[article]{aaltoseries}
\usepackage[utf8]{inputenc}


\begin{document}
 
%=========================================================

\title{Leveraging Function-as-a-Service (FaaS) for Edge Computing}

\author{Adriaan Knapen% Your first and last name: do _not_ add your student number
\\\textnormal{\texttt{adriaan.knapen@aalto.fi}}} % Your Aalto e-mail address

\affiliation{\textbf{Tutor}: Gopika Premsankar} % First and last name of your tutor

\maketitle

%==========================================================

\begin{abstract}

\vspace{3mm}
\noindent KEYWORDS: Edge Computing, Function-as-a-Service, FaaS, Serverless, Cloud Computing

\end{abstract}


%============================================================
\emph{Note: Citations are for now merely for my own reference.}

\section{Introduction} 
The emergence of cloud computing has been an important factor in the growing success of the Internet of Things (IoT) \cite{shi_promise_2016}.
Mainly because the cloud allows vast amounts of data to be processed efficiently \cite{shi_promise_2016}.
Something which is not attainable when processing the data directly on the IoT devices  \cite{shi_promise_2016}.
Although IoT has greatly benefited from cloud solutions is it not the silver bullet. % Sentence?
Using the cloud as the data processing backend for IoT induces several problems, like high latency and bandwidth constraints \cite{shi_edge_2016}.
One of the methods proposed to address this issue is \emph{edge computing} \cite{shi_edge_2016}.
Edge computing places computation resources close to the edge of the network \cite{shi_promise_2016}.
These computing resources are called \emph{edge devices} and potentially consist of various different types of devices, ranging from devices with low computation power, like smartphones, to computationally intense specialized micro-datacenters \cite{shi_promise_2016}.
Often is it not sufficient to merely employ edge devices, because there is no central node which has a global view of all employed edge nodes.
The necessity of a globally accessible system for all edge devices often comes from the demand for global data mining \cite{bonomi_fog_2012}.
One method of resolving this necessity is using cloud solutions.
Combining edge and cloud computing has shown to be advantageous, results show that both execution speed could be increased and energy consumption could be decreased by a factor 20 \cite{chun_clonecloud:_2011}.

Within the wide range of cloud solutions is the serverless paradigm, also known as Function-as-a-Service (FaaS), gaining rapid adoption \cite{.

%============================================================
\section{Methodology}

%============================================================
\section{Results}



%============================================================
\section{Discussion}


%============================================================
\section{Conclusion}



%============================================================


\bibliographystyle{plain}
\bibliography{edge-computing}

\end{document}
